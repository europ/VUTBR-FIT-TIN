\documentclass[11pt,a4paper]{article}

\usepackage[left=2cm,text={17cm,24cm},top=3cm]{geometry}
\usepackage[slovak]{babel}
\usepackage[utf8]{inputenc}
\usepackage[T1]{fontenc}

\usepackage{url}
\usepackage{enumerate}

\usepackage{float}
\usepackage{xcolor}
\usepackage{siunitx}
\usepackage{amsmath}
\usepackage{accents}
\usepackage{listings}
\usepackage{csquotes}
\usepackage{hyperref}
\usepackage{textcomp}
\usepackage{amsfonts}
\usepackage{breakurl}
\usepackage{etoolbox}
\usepackage{graphicx}
\usepackage{multicol}
\usepackage{multirow}
\usepackage{supertabular}
\usepackage{tikz}
\usepackage[titles]{tocloft}

\renewcommand{\cftdot}{}
\newcommand{\red}[1]{\textcolor{red}{#1}}
\newcommand{\blue}[1]{\textcolor{blue}{#1}}

\definecolor{OliveGreen}{rgb}{0,0.7,0}
\newcommand{\green}[1]{\textcolor{OliveGreen}{#1}}
\newcommand{\TODO}{\textbf{\textcolor{red}{TODO}}} % red bold TODO
\newcommand{\D}{\Delta}
\newcommand{\EOT}{\Delta^\omega} % end of tape
\newcommand{\UL}[1]{\underbar{$#1$}} % underline

\setlength\parindent{0pt}

\def\UrlBreaks{\do\/\do-}
\newcommand{\tilda}{\raisebox{0.5ex}{\texttildelow}}

\graphicspath{{.}}
\patchcmd{\thebibliography}{\section*{\refname}}{}{}{}

\begin{document}

\begin{titlepage}
    \begin{center}
        \Huge
        \textsc{
            Fakulta informačních technologií\\
            Vysoké učení technické v~Brně
        }
        \vspace{80px}
        \begin{figure}[!h]
            \centering
            \includegraphics[scale=0.3]{img/vutbr-fit-logo.eps}
        \end{figure}
        \\[15mm]
        \Huge{
            \textbf{
                TIN
            }
        }
        \\[1.5mm]
        \huge{
            \textbf{
                Teoretická informatika
            }
        }
        \\[2.5em]
        \LARGE{
            \textbf{
                3. domáca úloha
            }
        }
        \vfill
    \end{center}
        \Large{
            Adrián Tóth (xtotha01)\hfill \today
        }
\end{titlepage}

\setlength{\parskip}{0pt}
\hypersetup{hidelinks}\tableofcontents
\setlength{\parskip}{0pt}

\newpage
\section{Príklad číslo 1}

\subsection{(a)}

Pre $f(0)$ je reťazec $x$ prázdny pre ktorý páska 4 obsahuje výslednú hodnotu $1$ t.j. $f(0)=1$.\\[-0.5em]

\begin{tabular}{r|l|l|l}
    &                    & \tiny{$R^41^4L^4$}   & \tiny{$R^1$} \\\hline
  1 & $\UL{\D} \EOT$ & $\UL{\D} \EOT$   & $\D \UL{\D} \EOT$    \\
  2 & $\UL{\D} \EOT$ & $\UL{\D} \EOT$   & $\UL{\D} \EOT$       \\
  3 & $\UL{\D} \EOT$ & $\UL{\D} \EOT$   & $\UL{\D} \EOT$       \\
  4 & $\UL{\D} \EOT$ & $\UL{\D} 1 \EOT$ & $\UL{\D} 1 \EOT$     \\
\end{tabular}

\hfill\\[-1em]

Pre $f(1)$ je reťazec $x = 1$ pre ktorý páska 4 obsahuje výslednú hodnotu $1$ t.j. $f(1)=1$.\\[-0.5em]

\begin{tabular}{r|l|l|l|l|l|l|l|l|l}
    &                  &\tiny{$R^41^4L^4$}& \tiny{$R^1$}     & \tiny{$CP(3,2)$}  & \tiny{$L^3_\D$}   & \tiny{$CP(4,3)$}    &\tiny{$L^2_\D L^3_\D L^4_\D$}& \tiny{$CP(2,4)L^4$} & \\\hline
  1 & $\UL{\D} 1 \EOT$ & $\UL{\D} 1 \EOT$ & $\D \UL{1} \EOT$ & $\D \UL{1} \EOT$  & $\D \UL{1} \EOT$  & $\D \UL{1} \EOT$    & $\D \UL{1} \EOT$            & $\D \UL{1} \EOT$    & \\
  2 & $\UL{\D} \EOT$   & $\UL{\D} \EOT$   & $\UL{\D} \EOT$   & $\D \UL{\D} \EOT$ & $\D \UL{\D} \EOT$ & $\D \UL{\D} \EOT$   & $\UL{\D} \EOT$              & $\D \UL{\D} \EOT$   & ... \\
  3 & $\UL{\D} \EOT$   & $\UL{\D} \EOT$   & $\UL{\D} \EOT$   & $\D \UL{\D} \EOT$ & $\UL{\D} \EOT$    & $\D 1 \UL{\D} \EOT$ & $\UL{\D} 1 \EOT$            & $\UL{\D} 1 \EOT$    & \\
  4 & $\UL{\D} \EOT$   & $\UL{\D} 1 \EOT$ & $\UL{\D} 1 \EOT$ & $\UL{\D} 1 \EOT$  & $\UL{\D} 1 \EOT$  & $\D 1 \UL{\D} \EOT$ & $\UL{\D} 1 \EOT$            & $\UL{\D} 1 \EOT$    & \\
\end{tabular}

\begin{flushright}
  \begin{tabular}{r|l|l|l}
        &\tiny{$CP(3,4)$}    & \tiny{$L^2_\D L^3_\D L^4_\D$} & \tiny{$R^1$}        \\\hline
        &$\D \UL{1} \EOT$    & $\D \UL{1} \EOT$              & $\D 1 \UL{\D} \EOT$ \\
    ... &$\D \UL{\D} \EOT$   & $\UL{\D} \EOT$                & $\UL{\D} \EOT$      \\
        &$\D 1 \UL{\D} \EOT$ & $\UL{\D} 1 \EOT$              & $\UL{\D} 1 \EOT$    \\
        &$\D 1 \UL{\D} \EOT$ & $\UL{\D} 1 \EOT$              & $\UL{\D} 1 \EOT$    \\
  \end{tabular}
\end{flushright}

\hfill\\[-1em]

Pre $f(2)$ je reťazec $x = 11$ pre ktorý páska 4 obsahuje výslednú hodnotu $11$ t.j. $f(2)=2$.\\[-0.5em]

\begin{tabular}{r|l|l|l|l|l|l|l|l|l}
    &                    &\tiny{$R^41^4L^4$}  & \tiny{$R^1$}       & \tiny{$CP(3,2)$}   & \tiny{$L^3_\D$}    & \tiny{$CP(4,3)$}    &\tiny{$L^2_\D L^3_\D L^4_\D$}& \tiny{$CP(2,4)L^4$} & \\\hline
  1 & $\UL{\D} 1 1 \EOT$ & $\UL{\D} 1 1 \EOT$ & $\D \UL{1} 1 \EOT$ & $\D \UL{1} 1 \EOT$ & $\D \UL{1} 1 \EOT$ & $\D \UL{1} 1 \EOT$  & $\D \UL{1} 1 \EOT$          & $\D \UL{1} 1 \EOT$  & \\
  2 & $\UL{\D} \EOT$     & $\UL{\D} \EOT$     & $\UL{\D} \EOT$     & $\D \UL{\D} \EOT$  & $\D \UL{\D} \EOT$  & $\D \UL{\D} \EOT$   & $\UL{\D} \EOT$              & $\D \UL{\D} \EOT$   & ... \\
  3 & $\UL{\D} \EOT$     & $\UL{\D} \EOT$     & $\UL{\D} \EOT$     & $\D \UL{\D} \EOT$  & $\UL{\D} \EOT$     & $\D 1 \UL{\D} \EOT$ & $\UL{\D} 1 \EOT$            & $\UL{\D} 1 \EOT$    & \\
  4 & $\UL{\D} \EOT$     & $\UL{\D} 1 \EOT$   & $\UL{\D} 1 \EOT$   & $\UL{\D} 1 \EOT$   & $\UL{\D} 1 \EOT$   & $\D 1 \UL{\D} \EOT$ & $\UL{\D} 1 \EOT$            & $\UL{\D} 1 \EOT$    & \\
\end{tabular}

\begin{center}
  \begin{tabular}{r|l|l|l|l|l|l|l}
        &\tiny{$CP(3,4)$}    & \tiny{$L^2_\D L^3_\D L^4_\D$} & \tiny{$R^1$}       & \tiny{$CP(3,2)$}    & \tiny{$L^3_\D$}     & \tiny{$CP(4,3)$}    & \\\hline
        &$\D \UL{1} 1 \EOT$  & $\D \UL{1} 1 \EOT$            & $\D 1 \UL{1} \EOT$ & $\D 1 \UL{1} \EOT$  & $\D 1 \UL{1} \EOT$  & $\D 1 \UL{1} \EOT$  & \\
    ... &$\D \UL{\D} \EOT$   & $\UL{\D} \EOT$                & $\UL{\D} \EOT$     & $\D 1 \UL{\D} \EOT$ & $\D 1 \UL{\D} \EOT$ & $\D 1 \UL{\D} \EOT$ & ... \\
        &$\D 1 \UL{\D} \EOT$ & $\UL{\D} 1 \EOT$              & $\UL{\D} 1 \EOT$   & $\D 1 \UL{\D} \EOT$ & $\UL{\D} 1 \EOT$    & $\D 1 \UL{\D} \EOT$ & \\
        &$\D 1 \UL{\D} \EOT$ & $\UL{\D} 1 \EOT$              & $\UL{\D} 1 \EOT$   & $\UL{\D} 1 \EOT$    & $\UL{\D} 1 \EOT$    & $\D 1 \UL{\D} \EOT$ & \\
  \end{tabular}
\end{center}

\begin{flushright}
  \begin{tabular}{r|l|l|l|l|l}
        & \tiny{$L^2_\D L^3_\D L^4_\D$} & \tiny{$CP(2,4)L^4$} & \tiny{$CP(3,4)$}      & \tiny{$L^2_\D L^3_\D L^4_\D$} & \tiny{$R^1$}          \\\hline
        & $\D 1 \UL{1} \EOT$            & $\D 1 \UL{1} \EOT$  & $\D 1 \UL{1} \EOT$    & $\D 1 \UL{1} \EOT$            & $\D 1 1 \UL{\D} \EOT$ \\
    ... & $\UL{\D} 1 \EOT$              & $\D 1 \UL{\D} \EOT$ & $\D 1 \UL{\D} \EOT$   & $\UL{\D} 1 \EOT$              & $\UL{\D} 1 \EOT$      \\
        & $\UL{\D} 1 \EOT$              & $\UL{\D} 1 \EOT$    & $\D 1 \UL{\D} \EOT$   & $\UL{\D} 1 \EOT$              & $\UL{\D} 1 \EOT$      \\
        & $\UL{\D} 1 \EOT$              & $\D \UL{1} \EOT$    & $\D 1 1 \UL{\D} \EOT$ & $\UL{\D} 1 1 \EOT$            & $\UL{\D} 1 1 \EOT$    \\
  \end{tabular}
\end{flushright}

\hfill\\[-1em]

Pre $f(0),f(1),f(2),f(3),f(4),f(5)$ odpovedá rada čísel $1,1,2,3,5,8$. Na základe získaných hodnôt a faktu, že sa jedná o veľmi známu radu čísel vyplýva, že funkcia~$f$ generuje čísla z Fibonacciho rady.

\newpage
\subsection{(b)}

Parciálne rekurzívnu funckiu $f$ môžeme definovať ako:

\begin{center}
\begin{tabular}{l}
$f(0) = 1$ \\
$f(x+1) = plus(f(x), h(x))$ \\
$h(x)=f(monus(x,1))$ \\
\end{tabular}
\end{center}

\newpage
\section{Príklad číslo 2}

...

\newpage
\section{Príklad číslo 3}

...

\newpage
\section{Príklad číslo 4}

...

\newpage
\section{Príklad číslo 5}

...\cite{TIN}

\newpage
\section{Literatúra}
\bibliographystyle{slovakiso}
\begin{flushleft}
    \bibliography{quotation}
\end{flushleft}

\end{document}

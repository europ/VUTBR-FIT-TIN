\documentclass[11pt,a4paper]{article}

\usepackage[left=2cm,text={17cm,24cm},top=3cm]{geometry}
\usepackage[slovak]{babel}
\usepackage[utf8]{inputenc}
\usepackage[T1]{fontenc}

\usepackage{url}
\usepackage{enumerate}

\usepackage{float}
\usepackage{xcolor}
\usepackage{siunitx}
\usepackage{listings}
\usepackage{csquotes}
\usepackage{hyperref}
\usepackage{textcomp}
\usepackage{amsfonts}
\usepackage{breakurl}
\usepackage{etoolbox}
\usepackage{graphicx}
\usepackage{multicol}
\usepackage{multirow}
\usepackage{supertabular}
\usepackage{tikz}
\usepackage[titles]{tocloft}

\renewcommand{\cftdot}{}
\newcommand{\red}[1]{\textcolor{red}{#1}}

\setlength\parindent{0pt}

\def\UrlBreaks{\do\/\do-}
\newcommand{\tilda}{\raisebox{0.5ex}{\texttildelow}}

\graphicspath{{.}}
\patchcmd{\thebibliography}{\section*{\refname}}{}{}{}

\begin{document}

\begin{titlepage}
    \begin{center}
        \Huge
        \textsc{
            Fakulta informačních technologií\\
            Vysoké učení technické v~Brně
        }
        \vspace{80px}
        \begin{figure}[!h]
            \centering
            \includegraphics[scale=0.3]{img/vutbr-fit-logo.eps}
        \end{figure}
        \\[15mm]
        \Huge{
            \textbf{
                TIN
            }
        }
        \\[1.5mm]
        \huge{
            \textbf{
                Teoretická informatika
            }
        }
        \\[2.5em]
        \LARGE{
            \textbf{
                2. domáca úloha
            }
        }
        \vfill
    \end{center}
        \Large{
            Adrián Tóth (xtotha01)\hfill \today
        }
\end{titlepage}

\setlength{\parskip}{0pt}
\hypersetup{hidelinks}\tableofcontents
\setlength{\parskip}{0pt}


\newpage
\section{Príklad číslo 1} %############################################################################

\subsection{(a)}\label{sec:a}
\textit{Definice 4.29}~\cite{TIN}(str. č. 97) Označme $ZAV_n$ pre $n \geq 0$ jazyky setávající ze všech vyvážených řetězců závorek $n$ typů. Tyto jazyky -- označované též jako Dyckovy jazyky -- jsou generovány gramatikami s pravidly tvaru: $S \rightarrow [^{1}\ S\ ]^{1} \ |\  [^{2}\ S\ ]^{2} \ |\  ... \ |\  [^{n}\ S\ ]^{n} | \ SS \ | \ \varepsilon $

\rule{17cm}{0.4pt}

\hfill\\[-2em]

Z hore uvedenej definície pre náš príklad vyplýva, že náš Dyckov jazyk $L$ je generovaný gramatikou

\begin{center}
$S \rightarrow \varepsilon \ | \ SS \ | \ [\ S \ ]$
\end{center}

ktorá obsahuje iba jeden typ zátvoriek ktorými sú $[$ a $]$.\\

Pre každé slovo $w \in L$, pre ktoré platí že $w \neq \varepsilon$, muselo byť aspoň raz použité pravidlo $S \rightarrow [\ S \ ]$ v derivácii. Na základe tohto, vieme určiť najkratší neprázdny prefix slova $w$ patriaci do $L$ ktorým je $[\ ]$. S využitím pravidla $S \rightarrow [\ S \ ]$ vygenerujeme jeden pár zátvoriek, pričom medzi zátvorkami sa nachádza neterminál $S$, ktorým je možné ďalej aplikovať ďalšie pravidlá a generovať reťazec $u$ -- presnejšie reťazec $u$ patriaci do jazyka $L$. V prípade použitia pravidla $S \rightarrow SS$ pred pravidlom $S \rightarrow [\ S \ ]$ vieme generovať ďalší reťazec na pravej strane čo odpovedá reťazcu $v$ ktorý patrí do $L$, t.j. vieme generovať za $[u]$ reťazec $v$ patriaci do jazyka $L$.\\

Takže, každé slovo $w \in L$, pre ktoré platí že $w \neq \varepsilon$, vieme zapísať v tvare $[u]v$ kde $u,v \in L$ pretože

\begin{center}
$S \Rightarrow SS \Rightarrow [S]S \Rightarrow^{*} [u]v$
\end{center}

Keďže $S \Rightarrow^* u$ a $S \Rightarrow^* v$ ktoré patria do jazyka $L$, tak $S \Rightarrow^* [u]v$ tiež patrí do jazyka, keďže $u, v \in L$ a $[,] \in L$. Je zrejmé, že ak $S \Rightarrow^* w$ a $S \Rightarrow^* [u]v$ tak potom platí že $S \Rightarrow^* w = [u]v$ t.j. $w = [u]v$.\\

\subsection{(b)}

\textbf{Báza}
\begin{flushright}
\begin{minipage}{0.95\textwidth}
    $\varepsilon \in L$ pretože $S \Rightarrow^{*} \varepsilon$ keďže existuje pravidlo $S \rightarrow \varepsilon$\\
\end{minipage}
\end{flushright}

\textbf{Indukčný predpoklad}
\begin{flushright}
\begin{minipage}{0.95\textwidth}
    $S \Rightarrow^{*} w$ kde $w \in L \wedge w = [u]v$ čo platí pre $j < i$\\
\end{minipage}
\end{flushright}


\underline{\textbf{Pre i + 1}}
\begin{flushright}
\begin{minipage}{0.95\textwidth}
    $S \Rightarrow^{*} w$ kde $w \in L$ pre ktoré platí, že $\#_{[}(w)=i+1$.\\
    Z definicie Dyckovho jazyka platí, že $\#_{]}(w)=i+1$.\\
    Potom vieme $w$ zapísať ako $w=[u]v$ podľa bodu (a)(kapitola \ref{sec:a}) $\Longrightarrow$ $\#_{[}(u) + \#_{[}(v) = i$.\\
    Analogicky musí platiť $\#_{]}(u) + \#_{]}(v) = i$.
\end{minipage}
\end{flushright}

\hfill\\
\underline{\textbf{Pre i}}
\begin{flushright}
\begin{minipage}{0.95\textwidth}

    $\#_{[}(u) + \#_{[}(v) \stackrel{\text{?}}{=} i$ analogicky pre $\#_{]}(u) + \#_{]}(v) \stackrel{\text{?}}{=} i$\\

    % 1)
    \textbf{1.)} $\#_{[}(u) = 0 \Rightarrow \#_{[}(v) = i \quad \vee \quad \#_{[}(v) = 0 \Rightarrow \#_{[}(u) = i$\\[-1em]
    \begin{flushright}
    \begin{minipage}{0.95\textwidth}
        Ak je buď $\#_{[}(u)$ alebo buď $\#_{[}(v)$ rovné nule, tak ho vieme vygenerovať z pravidla $S$ na základe indukčnej bázi.\\

        Ak je buď $\#_{[}(u)$ alebo buď $\#_{[}(v)$ rovné $i$, tak ten prvok prepíšeme pomocou vzorca

        \begin{center}
        $w'=[u']v' \Rightarrow \#_{[}(u') + \#_{[}(v') = i-1$, analogicky $\#_{]}(u') + \#_{]}(v') = i-1$
        \end{center}

        Na základe indukčného predpokladu vieme z $S$ vygenerovať $u'$ a $v'$.
        \begin{center}
        $S \Rightarrow [S]S \Rightarrow^{*} [u']v' = w'$ kde $\#_{[}(u') + \#_{[}(v') = i$, analogicky $\#_{]}(u') + \#_{]}(v') = i$.
        \end{center}
    \end{minipage}
    \end{flushright}

    \hfill\\[-1em]

    % 2)
    \textbf{2.)} $\#_{[}(u) \neq 0 \quad \wedge \quad \#_{[}(v) \neq 0 \quad \wedge \quad \#_{[}(u)+\#_{[}(v) \leq i$\\[-1em]
    \begin{flushright}
    \begin{minipage}{0.95\textwidth}
        Keď $\#_{[}(u)$ a $\#_{[}(v)$ sú nenulové, tak musí platiť že
        \begin{center}
        $\#_{[}(u) < i \wedge \#_{[}(v) < i$ \ t.j. \ $\#_{[}(u) = i - M \wedge \#_{[}(v) = i - N$ kde $M,N \in \mathbb{N} \setminus \{0\}$
        \end{center}
    \end{minipage}
    \end{flushright}

\end{minipage}
\end{flushright}
\hfill\\

$S \Rightarrow [S]S \Rightarrow^{*} [u]v = w$ kde $\#_{[}(u) + \#_{[}(v) = i+1$, analogicky $\#_{]}(u) + \#_{]}(v) = i+1$.


\newpage
\section{Príklad číslo 2} %############################################################################

\textit{Veta 4.19}~\cite{TIN}(str. č. 92): Nechť $L$ je bezkontextový jazyk. Pak existuje konstanta $k>0$ taková že je-li $z \in L$ a $|z| \geq k$, pak lze $z$ napsat ve tvaru:

\begin{center}
$z = uvwxy, vx \neq \varepsilon, |vwx| \leq k$
\end{center}

a pro všechna $i \geq 0$ je $uv^{i}wx^{i}y \in L$.

\rule{17cm}{0.4pt}

\hfill\\[-2em]

Nech $L_{primes}$ je bezkontextový jazyk.\\

Tak existuje celočíselná konštanta $k > 0$ taká, že ak $z \in L$ a $|z| \geq k$, tak

\begin{center}
$z = uvwxy \wedge vx \neq \varepsilon \wedge |vwx| \leq k \wedge uv^{i}wx^{i}y \in L$ kde $i \geq 0$\\
\end{center}

Zvoľme prvočíslo $r$ väčšie ako ako $k$ t.j. $r \geq k$ kde $r$ je prvočíslo.\\

Potom platí, že

\begin{center}
$a^{r} \in L \wedge |a^{r}| = r$ kde $r \geq p \ \Longrightarrow \ a^{r} = uvwxy \wedge vx \neq \varepsilon \wedge |vwx| \leq k \wedge uv^{i}wx^{i}y \in L$ pre $i \geq 0$\\
\end{center}

Nech
\begin{center}
$v = a^{M} \Rightarrow |v| = M$\\
$x = a^{N} \Rightarrow |x| = N$\\
$w = a^{O} \Rightarrow |w| = O$\\
\end{center}

Tak musí platiť že $M+N > 0$ pretože $vx \neq \varepsilon$ a $k \geq M+N+O$ pretože $|vwx| \leq k$.\\

Zvoľme $i=r+1$, potom

\begin{center}
$uv^{r+1}wx^{r+1}y \in L$\\[0.5em]
$|uv^{r+1}wx^{r+1}y| = |uvwxy| + |v^{r}| + |x^{r}| = r + r \cdot M + r \cdot N = r \cdot (1+M+N)$ čo nie je prvočíšlo\\
\end{center}

A z toho vyplýva spor pretože

\begin{center}
$uv^{r+1}wx^{r+1}y \notin L$
\end{center}

Takže jazyk $L_{primes}$ nie je bezkontextový jazyk.

\newpage
\section{Príklad číslo 3} %############################################################################

\subsection{Nerozhodnuteľnosť}

Problém môžeme charakterizovať jazykom $L$ pre ktorý platí

\begin{center}
    $L = \{ \ \langle M_{L} \rangle \ | \ M_{L} \text{ je } TS: \exists w \in \text{\textit{Affine}}: w \in L(M_{L}) \ \}$
\end{center}

Problém členstva je charakterizovaný jazykom $MP$ pre ktorý platí

\begin{center}
    $MP = \{ \ \langle M_{MP} \rangle \# w  \ | \ M_{MP} \text{ je } TS \text{ ktorý prijme } w \ \}$
\end{center}

Zostavíme redukciu

\begin{center}
    $\sigma: \{0,1,\#\}^{*} \longrightarrow \{0,1\}^{*}$ z jazyka $MP$ na $L$
\end{center}

$TS \ M_{\sigma}$ implementujúci $\sigma$ priradí každému vstupu $x \in \{0,1,\#\}^{*}$ reťazec $\langle M_{x} \rangle$, kde $M_{x}$ je $TS$, ktorý na vstupu $y \in \{0,1\}^{*}$ pracuje následovne:

\begin{enumerate}
    \item $M_{x}$ zmaže svoj vstup $y$.
    \item Zapíše na pásku reťazec $x$.
    \item $M_{x}$ posúdi, zda $x=x_{1}\#x_{2}$ pre $x_1$, ktorý je kódom $TS$, a $x_2$, ktorý je kódom jeho vstupu. Pokiaľ nie, odmietne.
    \item Inak $M_{x}$ simuluje činnosť $TS$ s kódom $x_1$ na reťazci s kódom $x_2$.
        \begin{itemize}
            \item Ak $x_1$ prijme $x_2$, tak $M_x$ prijme.
            \item Ak $x_1$ odmietne $x_2$, tak $M_x$ odmietne.
            \item Inak cyklí.
        \end{itemize}
\end{enumerate}

$M_{\sigma}$ je možné implementovať úplným $TS$. Konečne tento $TS$ vypíše kód $M_x$, ktorý sa skladá zo štyroch komponent, ktoré odpovedajú vyššie uvedeným krokom. Tri z nich sú pritom konštantné (nezávisia na $x$) -- konkrétne (1) zmazanie pásky, (2) test na dobré sformovanie instancie $MP$ a (3) simulácia daného $TS$ na danom vstupe (pomocou úplného $TS$). $TS$ implementujúci tieto kroky, ktoré evidentne existujú, môžeme pripraviť vopred a $M_{\sigma}$ vypíše kód spolu s kódom na predanie riadenia. Zostáva vygenerovať kód $TS$, ktorý zapíše na pásku dané $x = a_1a_2...a_n$. To je možné ale ľahko realizovať pomocou $TS$ $Ra_1Ra_2R...Ra_n$. \\

Skúmajme možné jazyky $TS$ $M_x$:

\begin{itemize}
    \item $L(M_x) = \emptyset \Longleftrightarrow $ ($x$ nie je správne sformovaná instancia $MP$) alebo ($x=x_1\#x_2$ a $TS$ s kódom $x_1$ na reťazci s kódom $x_2$ odmietne) alebo ($x=x_1\#x_2$ a $TS$ s kódom $x_1$ na reťazci s kódom $x_2$ neskončí t.j. cyklí)
    \item $L(M_x) = \Sigma^* \Longleftrightarrow $ ($x$ je správne sformovaná instancia $MP$, kde $x=x_1\#x_2$ a $TS$ s kódom $x_1$ na reťazci s kódom $x_2$ prijme)
\end{itemize}

Ak $L(M_x) = \Sigma^*$ je zrejmé, že jazyk $L(M_x)$ iste obsahuje aspoň jeden reťazec ktorý patrí do jazyka \textit{Affine}.\\

Teraz už ľahko ukážeme, že $\sigma$ zachováva členstvo $\langle M_{x} \rangle \in L \Leftrightarrow L(M_x) = \Sigma^{*} \Leftrightarrow x=x_1\#x_2$ kde $x_1$ je kód $TS$, ktorý zastaví na vstupe s kódem $x_2 \Leftrightarrow x \in MP$.


\newpage
\subsection{Čiastočná rozhodnuteľnosť}

...

\newpage
\section{Príklad číslo 4} %############################################################################

...


\newpage
\section{Literatúra} %#################################################################################

\bibliographystyle{slovakiso}
\begin{flushleft}
    \bibliography{quotation}
\end{flushleft}

\end{document}

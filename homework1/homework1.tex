%%%%%%%%%%%%%%%%%%%%%%%%%%%%%%%%%%%%%%%%%%%%%%%%%%%%%%%%%%%%%%%%%%%%%%
% Date:
%     21.10.2018
%
% Course:
%     TIN - Intelligent Sensors
%     https://www.fit.vutbr.cz/study/courses/index.php.en?id=12941
%
% Project:
%     1. homework
%
% Authors:
%     Adrián Tóth, xtotha01@stud.fit.vutbr.cz
%%%%%%%%%%%%%%%%%%%%%%%%%%%%%%%%%%%%%%%%%%%%%%%%%%%%%%%%%%%%%%%%%%%%%%

\documentclass[11pt,a4paper]{article}

\usepackage[left=2cm,text={17cm,24cm},top=3cm]{geometry}
\usepackage[slovak]{babel}
\usepackage[utf8]{inputenc}
\usepackage[T1]{fontenc}

\usepackage{url}
\usepackage{enumerate}

\usepackage{float}
\usepackage{xcolor}
\usepackage{siunitx}
\usepackage{listings}
\usepackage{csquotes}
\usepackage{hyperref}
\usepackage{textcomp}
\usepackage{breakurl}
\usepackage{etoolbox}
\usepackage{graphicx}
\usepackage{multicol}
\usepackage{multirow}
\usepackage{supertabular}
\usepackage{tikz}
\usepackage[titles]{tocloft}

\renewcommand{\cftdot}{}
\newcommand{\red}[1]{\textcolor{red}{#1}}

\setlength\parindent{0pt}

\def\UrlBreaks{\do\/\do-}
\newcommand{\tilda}{\raisebox{0.5ex}{\texttildelow}}

\graphicspath{{.}}
\patchcmd{\thebibliography}{\section*{\refname}}{}{}{}

\begin{document}

\begin{titlepage}
    \begin{center}
        \Huge
        \textsc{
            Fakulta informačních technologií\\
            Vysoké učení technické v~Brně
        }
        \vspace{80px}
        \begin{figure}[!h]
            \centering
            \includegraphics[scale=0.3]{img/vutbr-fit-logo.eps}
        \end{figure}
        \\[15mm]
        \Huge{
            \textbf{
                TIN
            }
        }
        \\[1.5mm]
        \huge{
            \textbf{
                Teoretická informatika
            }
        }
        \\[2.5em]
        \LARGE{
            \textbf{
                1. domáca úloha
            }
        }
        \vfill
    \end{center}
        \Large{
            Adrián Tóth (xtotha01)\hfill \today
        }
\end{titlepage}

\setlength{\parskip}{0pt}
\hypersetup{hidelinks}\tableofcontents
\setlength{\parskip}{0pt}



\newpage
\section{Príklad číslo 1} %############################################################################

\subsection{(a)} %################################################################################

Vyjadríme si rozdiel množín ekvivalentným vzťahom pomocou prieniku a doplnku (komplementu), aby sme mohli využiť vetu zo študijného textu.
\begin{center}
$L_1 \setminus L_2 = L_1 \cap \overline{L_2}$
\end{center}

Podľa \textit{Vety 3.23}~\cite{TIN}(str. č. 50) platí, že trieda regulárnych jazykov $\mathcal{L}_3$ je uzavretá voči prieniku a doplnku (komplementu).

\begin{center}
$L_1,L_2 \in \mathcal{L}_3 \Rightarrow L_1 \cap \overline{L_2} \in \mathcal{L}_3$
\end{center}

Využitím hore uvedenej \textit{Vety 3.23} a vzťahov možeme stanoviť, že nasledujúci vzťah je platný.

\begin{center}
$L_1,L_2 \in \mathcal{L}_3 \Rightarrow L_1 \setminus L_2 \in \mathcal{L}_3$
\end{center}



\subsection{(b)} %################################################################################

Vyjadríme si rozdiel množín ekvivalentným vzťahom pomocou prieniku a doplnku (komplementu), aby sme mohli využiť vetu zo študijného textu.
\begin{center}
$L_1 \setminus L_2 = L_1 \cap \overline{L_2}$
\end{center}

Podľa \textit{Vety 4.27}~\cite{TIN}(str. č. 96) platí, že trieda deterministických bezkontextových jazykov $\mathcal{L}_2^D$ je uzavretá voči prieniku a doplnku (komplementu).

\begin{center}
$L_1 \in \mathcal{L}_3, L_2 \in \mathcal{L}_2^D \Rightarrow L_1 \cap \overline{L_2} \in \mathcal{L}_2^D$
\end{center}

Využitím hore uvedenej \textit{Vety 4.27} a vzťahov možeme stanoviť, že nasledujúci vzťah je platný.

\begin{center}
$L_1 \in \mathcal{L}_3, L_2 \in \mathcal{L}_2^D \Rightarrow L_1 \setminus L_2 \in \mathcal{L}_2^D$
\end{center}



\subsection{(c)} %################################################################################

Predpokladajme, že $L_1 \in \mathcal{L}_3, L_2 \in \mathcal{L}_2 \Rightarrow L_1 \setminus L_2 \in \mathcal{L}_2$ je pravdivý vzťah.\\

Ak berieme v úvahu, že $L_1 = \Sigma^*$ (regulárny jazyk), tak musí platiť:
\begin{center}
$\Sigma^* \setminus L_2 \in \mathcal{L}_2 \Rightarrow \overline{L_2} \in \mathcal{L}_2 \Rightarrow \underline{\textbf{SPOR}}$
\end{center}

Vznikol nám spor pri $\overline{L_2} \in \mathcal{L}_2$ z toho dôvodu, že podľa \textit{Vety 4.24}~\cite{TIN}(str. č. 95) platí, že bezkontextové jazyky nie sú uzavreté voči prieniku a doplnku.\\

Takže $L_1 \in \mathcal{L}_3, L_2 \in \mathcal{L}_2 \Rightarrow L_1 \setminus L_2 \in \mathcal{L}_2$ nie je pravdivý vzťah.




\newpage
\section{Príklad číslo 2} %############################################################################

$M_L = (Q,\Sigma,\Gamma,\delta,q_0,Z_0,F)$\\
\\
$Q = \{q_0,q_1,q_2,q_3\}$\\
$\Sigma = \{\#,0,1,2\}$\\
$\Gamma = \{Z_0,1\}$\\
$F = \{q_3\}$\\
$\delta$:\\[-2.9em]
\begin{center}
\begin{minipage}{0.9\textwidth}
$\delta(q_0,0,Z_0)           = (q_0,Z_0)$\\
$\delta(q_0,1,Z_0)           = (q_0,1Z_0)$\\
$\delta(q_0,2,Z_0)           = (q_0,11Z_0)$\\
$\delta(q_0,0,1)             = (q_0,1)$\\
$\delta(q_0,1,1)             = (q_0,11)$\\
$\delta(q_0,2,1)             = (q_0,111)$\\
$\delta(q_0,\#,1)            = (q_1,1)$\\
$\delta(q_0,\#,Z_0)          = (q_1,Z_0)$\\
$\delta(q_1,0,1)             = (q_1,1)$\\
$\delta(q_1,1,1)             = (q_1,\varepsilon)$\\
$\delta(q_1,2,1)             = (q_2,\varepsilon)$\\
$\delta(q_1,\varepsilon,Z_0) = (q_3,Z_0)$\\
$\delta(q_2,\varepsilon,1)   = (q_1,\varepsilon)$\\
$\delta(q_3,0,Z_0)           = (q_3,Z_0)$\\
\end{minipage}
\end{center}

\begin{center}
\begin{tikzpicture}[scale=0.2]
\tikzstyle{every node}+=[inner sep=0pt]
\draw [black] (34.3,-16.2) circle (3);
\draw (34.3,-16.2) node {$q_0$};
\draw [black] (34.3,-33.1) circle (3);
\draw (34.3,-33.1) node {$q_1$};
\draw [black] (51,-33.1) circle (3);
\draw (51,-33.1) node {$q_3$};
\draw [black] (51,-33.1) circle (2.4);
\draw [black] (17.5,-33.1) circle (3);
\draw (17.5,-33.1) node {$q_2$};
\draw [black] (34.3,-19.2) -- (34.3,-30.1);
\fill [black] (34.3,-30.1) -- (34.8,-29.3) -- (33.8,-29.3);
\draw (33.8,-24.65) node [right] {\begin{tabular}{c}$\#,1/1$\\$\#,Z_0/Z_0$\end{tabular}};
\draw [black] (37.3,-33.1) -- (48,-33.1);
\fill [black] (48,-33.1) -- (47.2,-32.6) -- (47.2,-33.6);
\draw (42.65,-32.6) node [above] {$\varepsilon,Z_0/Z_0$};
\draw [black] (31.757,-34.68) arc (-64.4656:-115.5344:13.588);
\fill [black] (20.04,-34.68) -- (20.55,-35.48) -- (20.98,-34.57);
\draw (25.9,-36.51) node [below] {$2,1/\varepsilon$};
\draw [black] (20.181,-31.764) arc (111.08444:68.91556:15.897);
\fill [black] (31.62,-31.76) -- (31.05,-31.01) -- (30.69,-31.94);
\draw (25.9,-30.2) node [above] {$\varepsilon,1/\varepsilon$};
\draw [black] (36.721,-34.852) arc (81.83507:-206.16493:2.25);
\draw (38.55,-39.57) node [below] {\begin{tabular}{c}$1,1/\varepsilon$\\$0,1/1$\end{tabular}};
\fill [black] (34.38,-36.09) -- (33.77,-36.81) -- (34.76,-36.95);
\draw [black] (51.702,-30.195) arc (194.13949:-93.86051:2.25);
\draw (56.43,-27.43) node [above] {$0,Z_0/Z_0$};
\fill [black] (53.73,-31.89) -- (54.63,-32.18) -- (54.39,-31.21);
\draw [black] (35.975,-13.725) arc (173.63503:-114.36497:2.25);
\draw (39.83,-9.64) node [right] {
\begin{tabular}{c}$0,Z_0/Z_0$\\$1,Z_0/1Z_0$\\$2,Z_0/11Z_0$\\$0,1/1$\\$1,1/11$\\$2,1/111$\end{tabular}};
\fill [black] (37.28,-16.02) -- (38.02,-16.61) -- (38.13,-15.62);
\draw [black] (28.6,-13.3) -- (31.63,-14.84);
\draw (27.91,-12.18) node [left] {$Z_0$};
\fill [black] (31.63,-14.84) -- (31.14,-14.03) -- (30.69,-14.92);
\end{tikzpicture}
\end{center}




\newpage
\section{Príklad číslo 3}
$L = \{ w_1\#w_2\ |\ w_1,w_2 \in \Sigma^*, \#_1(w_1) + (2 * \#_2(w_1)) = \#_1(w_2) + (2 * \#_2(w_2))\}$\\

\textit{Veta 3.18}~\cite{TIN}(str. č. 46): Nechť $L$ je nekonečný regulární jazyk. Pak existuje celočíselná konstanta $p>0$ taková, že platí: $w \in L \wedge |w| \geq p \Rightarrow w = xyz \wedge y \neq \varepsilon \wedge |xy| \leq p \wedge xy^iz \in L$ pro $i \geq 0$\\

Predpokladáme že jazyk L je regulárny jazyk a tak tento jazyk musí spĺňať hore uvedenú \textit{Vetu 3.18}.\\

Pre $w \in L: w=1^p\#1^p$ pre ktoré platí podmienka $|w| \geq p$ pretože platí $2p+1>p$, pričom z dôvodu podmienky $|xy| \leq p$ nastane jediný prípad a to:\\

$x=1^l \wedge y=1^m \wedge z=1^{p-l-m}\#1^p$ kde $l \geq 0$ a $m > 0 \wedge l+m \leq p$ pre $l,m \in N$\\

$xy^iz = 1^l(1^{m})^i1^{p-l-m}\#1^p = 1^{l+(i*m)+p-l-m}\#1^p = 1^{(i*m)+p-m}\#1^p \notin L$ pre všetky $i \neq 1$ pretože nesplňuje podmienku jazyka $L$.\\

Z predošlého vzťahu vyplýva, že jazyk $L$ nie je regulárny.



\newpage
\section{Príklad číslo 4}

\underline{\textbf{\large{ALGORITMUS}}}\\[0.3em]
\textbf{Vstup}: Pravá lineárna gramatika $G_P = (N,\Sigma,P,S)$\\
\textbf{Výstup}: Ľavá lineárna gramatika $G_L = (N',\Sigma',P',S')$ taká, že $L(G_P)=L(G_L)$\\
\textbf{Metóda}:\\[-2em]
\begin{center}
\begin{minipage}{0.9\textwidth}
\begin{enumerate}[1.)]
    \item $N' = N \cup \{S'\}$ kde $ S' \notin N$\\[-2em]
    \item $\Sigma' = \Sigma$\\[-2em]
    \item $P'$:\\[-3.275em]
    \begin{center}
    \begin{minipage}{0.85\textwidth}
            $\hspace{0.6em}\forall A,B,S \in N,\  w \in \Sigma^*:$\\
        \begin{tabular}{r|lcl}
            \textit{I.}   & $(S \rightarrow \varepsilon) \in P'$\\
            \textit{II.}  & $(B  \rightarrow Aw) \in P'$ & $\Longleftrightarrow$ & $(A \rightarrow wB) \in P$\\
            \textit{III.} & $(S' \rightarrow Aw) \in P'$ & $\Longleftrightarrow$ & $(A \rightarrow w ) \in P$\\
        \end{tabular}
    \end{minipage}
    \end{center}
\end{enumerate}
\end{minipage}
\end{center}

\hfill

\underline{\textbf{\large{DEMONŠTRÁCIA}}}\\[0.3em]
\textbf{Vstup}: Pravá lineárna gramatika $G = (\{S,A,B\}, \{a, b\}, P, S)$\\[-1.25em]
\begin{center}
\begin{tabular}{ll}
    P:&$S \rightarrow abA\ |\ bS$\\
    &$A \rightarrow bB\ |\ S\ |\ ab$\\
    &$B \rightarrow \varepsilon\ |\ aA$\\
\end{tabular}
\end{center}

\textbf{Realizácia}:\\[-2em]
\begin{center}
\begin{minipage}{0.9\textwidth}
\begin{enumerate}[1.)]
    \item $N' = N \cup \{S'\}$\\[-2em]
    \item $\Sigma' = \Sigma$\\[-2em]
    \item $P'$:\\[-3.45em]%[-3.275em]
    \begin{center}
    \begin{minipage}{0.85\textwidth}
            %$\hspace{0.6em}\forall A,B \in N,\  p \in \Sigma^*:$\\
        \begin{tabular}{lrlcl}
            $S   \rightarrow abA$         & sa transformuje na & $A  \rightarrow Sab$         & podľa & \textit{II.} \\
            $S   \rightarrow bS $         & sa transformuje na & $S  \rightarrow Sb $         & podľa & \textit{II.} \\
            $A   \rightarrow bB $         & sa transformuje na & $B  \rightarrow Ab $         & podľa & \textit{II.} \\
            $A   \rightarrow S  $         & sa transformuje na & $S  \rightarrow A  $         & podľa & \textit{II.} \\
            $A   \rightarrow ab $         & sa transformuje na & $S' \rightarrow Aab$         & podľa & \textit{III.} \\
            $B   \rightarrow \varepsilon$ & sa transformuje na & $S' \rightarrow B  $         & podľa & \textit{III.} \\
            $B   \rightarrow aA $         & sa transformuje na & $A  \rightarrow Ba $         & podľa & \textit{II.} \\
                                          & pridá sa           & $S  \rightarrow \varepsilon$ & podľa & \textit{I.} \\
        \end{tabular}
    \end{minipage}
    \end{center}
\end{enumerate}
\end{minipage}
\end{center}
\textbf{Výstup}: Ľavá lineárna gramatika $G_L = (N',\Sigma',P',S')$ taká, že $L(G)=L(G_L)$\\

\textbf{Overenie}:\\
Pravá lineárna gramatika $G_P$ derivuje reťazec $babbaabb$.
\begin{center}
$S \Rightarrow bS \Rightarrow babA  \Rightarrow babbB \Rightarrow babbaA \Rightarrow babbaS \Rightarrow babbaabA \Rightarrow babbaabbB \Rightarrow babbaabb$
\end{center}

Ľavá lineárna gramatika $G_L$ musí tiež derivovať reťazec $babbaabb$, keďže platí $L(G_P)=L(G_L)$.
\begin{center}
$S' \Rightarrow B \Rightarrow Ab \Rightarrow Sabb \Rightarrow Aabb \Rightarrow Baabb \Rightarrow Abaabb \Rightarrow Sabbaabb \Rightarrow Sbabbaabb \Rightarrow babbaabb$
\end{center}



\newpage
\section{Príklad číslo 5}

Definícia $\sim_L$ pre jazyk $L$:\\[-1.5em]
\begin{center}
$u \sim_L v \stackrel{\text{def}}{\Longleftrightarrow} (\#_a(u)mod\ 3 = \#_a(v)mod\ 3 \wedge ((\#_b(u) > 0 \wedge \#_b(v) > 0) \vee (\#_b(u) = 0 \wedge \#_b(v) = 0)))$
\end{center}

\hfill\\

Na základe \textit{Vety 3.21}~\cite{TIN}(str. č. 49), zostrojíme minimálny DKA na základe ktorého vieme určiť rozklad $\Sigma^* / \sim_L$.

\begin{center}
    \begin{tikzpicture}[scale=0.2]
        \tikzstyle{every node}+=[inner sep=0pt]
        \draw [black] (14.5,-12.7) circle (3);
        \draw (14.5,-12.7) node {$q_0$};
        \draw [black] (30.5,-12.7) circle (3);
        \draw (30.5,-12.7) node {$q_1$};
        \draw [black] (45.8,-12.7) circle (3);
        \draw (45.8,-12.7) node {$q_2$};
        \draw [black] (14.5,-27.2) circle (3);
        \draw (14.5,-27.2) node {$q_3$};
        \draw [black] (30.5,-27.2) circle (3);
        \draw (30.5,-27.2) node {$q_4$};
        \draw [black] (30.5,-27.2) circle (2.4);
        \draw [black] (45.8,-27.2) circle (3);
        \draw (45.8,-27.2) node {$q_5$};
        \draw [black] (45.8,-27.2) circle (2.4);
        \draw [black] (14.5,-15.7) -- (14.5,-24.2);
        \fill [black] (14.5,-24.2) -- (15,-23.4) -- (14,-23.4);
        \draw (14,-19.95) node [left] {$b$};
        \draw [black] (17.5,-27.2) -- (27.5,-27.2);
        \fill [black] (27.5,-27.2) -- (26.7,-26.7) -- (26.7,-27.7);
        \draw (22.5,-27.7) node [below] {$a$};
        \draw [black] (33.5,-27.2) -- (42.8,-27.2);
        \fill [black] (42.8,-27.2) -- (42,-26.7) -- (42,-27.7);
        \draw (38.15,-27.7) node [below] {$a$};
        \draw [black] (11.82,-28.523) arc (324:36:2.25);
        \draw (7.25,-27.2) node [left] {$b$};
        \fill [black] (11.82,-25.88) -- (11.47,-25) -- (10.88,-25.81);
        \draw [black] (30.5,-15.7) -- (30.5,-24.2);
        \fill [black] (30.5,-24.2) -- (31,-23.4) -- (30,-23.4);
        \draw (30,-19.95) node [left] {$b$};
        \draw [black] (31.823,-29.88) arc (54:-234:2.25);
        \draw (30.5,-34.45) node [below] {$b$};
        \fill [black] (29.18,-29.88) -- (28.3,-30.23) -- (29.11,-30.82);
        \draw [black] (48.48,-25.877) arc (144:-144:2.25);
        \draw (53.05,-27.2) node [right] {$b$};
        \fill [black] (48.48,-28.52) -- (48.83,-29.4) -- (49.42,-28.59);
        \draw [black] (6.9,-6.2) -- (12.22,-10.75);
        \fill [black] (12.22,-10.75) -- (11.94,-9.85) -- (11.29,-10.61);
        \draw [black] (16.157,-10.203) arc (141.61549:38.38451:17.851);
        \fill [black] (16.16,-10.2) -- (17.05,-9.89) -- (16.26,-9.27);
        \draw (30.15,-2.94) node [above] {$a$};
        \draw [black] (17.5,-12.7) -- (27.5,-12.7);
        \fill [black] (27.5,-12.7) -- (26.7,-12.2) -- (26.7,-13.2);
        \draw (22.5,-13.2) node [below] {$a$};
        \draw [black] (33.5,-12.7) -- (42.8,-12.7);
        \fill [black] (42.8,-12.7) -- (42,-12.2) -- (42,-13.2);
        \draw (38.15,-13.2) node [below] {$a$};
        \draw [black] (45.8,-15.7) -- (45.8,-24.2);
        \fill [black] (45.8,-24.2) -- (46.3,-23.4) -- (45.3,-23.4);
        \draw (45.3,-19.95) node [left] {$b$};
        \draw [black] (44.295,-29.791) arc (-35.11681:-144.88319:17.293);
        \fill [black] (16,-29.79) -- (16.06,-30.73) -- (16.87,-30.16);
        \draw (30.15,-37.64) node [below] {$a$};
    \end{tikzpicture}
\end{center}

Rozklad $\Sigma^* / \sim_L$ je tvorený nasledujúcimi šiestimi triedami:

\begin{center}
$L^{-1}(q_0) = \{w\ |\ \#_a(w) mod\ 3 = 0 \wedge \#_b(w) = 0\}$\\
$L^{-1}(q_1) = \{w\ |\ \#_a(w) mod\ 3 = 1 \wedge \#_b(w) = 0\}$\\
$L^{-1}(q_2) = \{w\ |\ \#_a(w) mod\ 3 = 2 \wedge \#_b(w) = 0\}$\\
$L^{-1}(q_3) = \{w\ |\ \#_a(w) mod\ 3 = 0 \wedge \#_b(w) > 0\}$\\
$L^{-1}(q_4) = \{w\ |\ \#_a(w) mod\ 3 = 1 \wedge \#_b(w) > 0\}$\\
$L^{-1}(q_5) = \{w\ |\ \#_a(w) mod\ 3 = 2 \wedge \#_b(w) > 0\}$\\
\end{center}

Rozklad $\Sigma^* / \sim_L$ má tak konečný počet tried (šesť tried t.j. relácia $\sim_L$ má index 6) z čoho vyplýva podľa \textit{Vety 3.20}~\cite{TIN}(str. č. 48), že sa jedná o regulárny jazyk t.j. platí \textit{Veta 3.21}~\cite{TIN}(str. č. 49), keďže sme vedeli zostrojiť minimálny konečný automat.\\

\hfill\\

Jazyk $L$ je tvorený zjednotením dvoch predošlých tried:

\begin{center}
$L = L^{-1}(q_4) \cup L^{-1}(q_5)$\\
\end{center}

\newpage
\section{Literatúra}
\bibliographystyle{slovakiso}
\begin{flushleft}
    \bibliography{quotation}
\end{flushleft}

\end{document}

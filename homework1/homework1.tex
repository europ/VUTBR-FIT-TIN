%%%%%%%%%%%%%%%%%%%%%%%%%%%%%%%%%%%%%%%%%%%%%%%%%%%%%%%%%%%%%%%%%%%%%%
% Date:
%     21.10.2018
%
% Course:
%     TIN - Intelligent Sensors
%     https://www.fit.vutbr.cz/study/courses/index.php.en?id=12941
%
% Project:
%     1. homework
%
% Authors:
%     Adrián Tóth, xtotha01@stud.fit.vutbr.cz
%%%%%%%%%%%%%%%%%%%%%%%%%%%%%%%%%%%%%%%%%%%%%%%%%%%%%%%%%%%%%%%%%%%%%%

\documentclass[11pt,a4paper]{article}

\usepackage[left=2cm,text={17cm,24cm},top=3cm]{geometry}
\usepackage[slovak]{babel}
\usepackage[utf8]{inputenc}
\usepackage[T1]{fontenc}

\usepackage{url}
\usepackage{enumerate}

\usepackage{float}
\usepackage{xcolor}
\usepackage{siunitx}
\usepackage{listings}
\usepackage{csquotes}
\usepackage{hyperref}
\usepackage{textcomp}
\usepackage{breakurl}
\usepackage{etoolbox}
\usepackage{graphicx}
\usepackage{multicol}
\usepackage{supertabular}
\usepackage{tikz}
\usepackage[titles]{tocloft}

\renewcommand{\cftdot}{}
\newcommand{\red}[1]{\textcolor{red}{#1}}

\setlength\parindent{0pt}

\def\UrlBreaks{\do\/\do-}
\newcommand{\tilda}{\raisebox{0.5ex}{\texttildelow}}

\graphicspath{{.}}
\patchcmd{\thebibliography}{\section*{\refname}}{}{}{}

\begin{document}

\begin{titlepage}
    \begin{center}
        \Huge
        \textsc{
            Fakulta informačních technologií\\
            Vysoké učení technické v~Brně
        }
        \vspace{80px}
        \begin{figure}[!h]
            \centering
            \includegraphics[scale=0.3]{img/vutbr-fit-logo.eps}
        \end{figure}
        \\[15mm]
        \Huge{
            \textbf{
                TIN
            }
        }
        \\[1.5mm]
        \huge{
            \textbf{
                Teoretická informatika
            }
        }
        \\[2.5em]
        \LARGE{
            \textbf{
                1. domáca úloha
            }
        }
        \vfill
    \end{center}
        \Large{
            Adrián Tóth (xtotha01)\hfill \today
        }
\end{titlepage}

\setlength{\parskip}{0pt}
\hypersetup{hidelinks}\tableofcontents
\setlength{\parskip}{0pt}

\newpage

\section{1. príklad} %############################################################################

\subsection{(a)} %################################################################################

Vyjadríme si rozdiel množín ekvivalentným vzťahom pomocou prieniku a doplnku (komplementu), aby sme mohli využiť vetu zo študijného textu.
\begin{center}
$L_1 \setminus L_2 = L_1 \cap \overline{L_2}$
\end{center}

Podľa \textit{Vety 3.23}~\cite{TIN}(str. č. 50) platí, že trieda regulárnych jazykov $\mathcal{L}_3$ je uzavretá voči prieniku a doplnku (komplementu).\\

Využitím hore uvedenej \textit{Vety 3.23} a vzťahu možeme stanoviť, že nasledujúci vzťah je platný.

\begin{center}
$L_1,L_2 \in \mathcal{L}_3 \Rightarrow L_1 \setminus L_2 \in \mathcal{L}_3$
\end{center}

\subsection{(b)} %################################################################################

Vyjadríme si rozdiel množín ekvivalentným vzťahom pomocou prieniku a doplnku (komplementu), aby sme mohli využiť vetu zo študijného textu.
\begin{center}
$L_1 \setminus L_2 = L_1 \cap \overline{L_2}$
\end{center}

Podľa \textit{Vety 4.27}~\cite{TIN}(str. č. 96) platí, že trieda deterministických bezkontextových jazykov $\mathcal{L}_2^D$ je uzavretá voči prieniku a doplnku (komplementu).\\

Využitím hore uvedenej \textit{Vety 4.27} a vzťahu možeme stanoviť, že nasledujúci vzťah je platný.

\begin{center}
$L_1 \in \mathcal{L}_3, L_2 \in \mathcal{L}_2^D \Rightarrow L_1 \setminus L_2 \in \mathcal{L}_2^D$
\end{center}


\subsection{(c)} %################################################################################

\red{
$L_1 \in \mathcal{L}_3, L_2 \in \mathcal{L}_2 \Rightarrow L_1 \setminus L_2 \in \mathcal{L}_2$\\
Dôkaz sporom (treba protipríklad):\\
$L \subseteq \Sigma^*$\\
$\overline{L} = \Sigma^* \setminus L$
$\Sigma^*$ je regulárny jazyk - $\mathcal{L}_2$
}

\newpage
\section{2. príklad} %############################################################################

$M_L = (Q,\Sigma,\Gamma,\delta,q_0,Z_0,F)$\\
\\
$Q = \{q_0,q_1,q_2,q_3\}$\\
$\Sigma = \{\#,0,1,2\}$\\
$\Gamma = \{Z_0,1\}$\\
$F = \{q_3\}$\\
$\delta$:\\
\begin{tabular}{l}
$\hspace{1.5em}\delta(q_0,0,Z_0)=\{(q_0,Z_0)\}$\\
$\hspace{1.5em}\delta(q_0,1,Z_0)=\{(q_0,1Z_0)\}$\\
$\hspace{1.5em}\delta(q_0,2,Z_0)=\{(q_0,11Z_0)\}$\\
$\hspace{1.5em}\delta(q_0,0,1)=\{(q_0,1)\}$\\
$\hspace{1.5em}\delta(q_0,1,1)=\{(q_0,11)\}$\\
$\hspace{1.5em}\delta(q_0,2,1)=\{(q_0,111)\}$\\
$\hspace{1.5em}\delta(q_0,\#,1)=\{(q_1,1)\}$\\
$\hspace{1.5em}\delta(q_0,\#,Z_0)=\{(q_1,Z_0)\}$\\
$\hspace{1.5em}\delta(q_1,2,1)=\{(q_2,\epsilon)\}$\\
$\hspace{1.5em}\delta(q_2,\epsilon,1)=\{(q_1,\epsilon)\}$\\
$\hspace{1.5em}\delta(q_1,1,1)=\{(q_1,\epsilon)\}$\\
$\hspace{1.5em}\delta(q_1,0,1)=\{(q_1,1)\}$\\
$\hspace{1.5em}\delta(q_1,0,Z_0)=\{(q_1,Z_0)\}$\\
$\hspace{1.5em}\delta(q_1,\epsilon,Z_0)=\{(q_3,\epsilon)\}$\\
\end{tabular}

\begin{center}
\begin{tikzpicture}[scale=0.2]
\tikzstyle{every node}+=[inner sep=0pt]
\draw [black] (34.5,-14.2) circle (3);
\draw (34.5,-14.2) node {$q_0$};
\draw [black] (34.5,-37.7) circle (3);
\draw (34.5,-37.7) node {$q_1$};
\draw [black] (52.2,-37.7) circle (3);
\draw (52.2,-37.7) node {$q_3$};
\draw [black] (52.2,-37.7) circle (2.4);
\draw [black] (16.4,-37.7) circle (3);
\draw (16.4,-37.7) node {$q_2$};
\draw [black] (25.8,-14.2) -- (31.5,-14.2);
\draw (25.3,-14.2) node [left] {$Z_0$};
\fill [black] (31.5,-14.2) -- (30.7,-13.7) -- (30.7,-14.7);
\draw [black] (37.18,-12.877) arc (144:-144:2.25);
\draw (41.75,-14.2) node [right] {\begin{tabular}{c}$0,Z_0/Z_0$\\$1,Z_0/1Z_0$\\$2,Z_0/11Z_0$\\$0,1/1$\\$1,1/11$\\$2,1/111$\end{tabular}};
\fill [black] (37.18,-15.52) -- (37.53,-16.4) -- (38.12,-15.59);
\draw [black] (34.5,-17.2) -- (34.5,-34.7);
\fill [black] (34.5,-34.7) -- (35,-33.9) -- (34,-33.9);
\draw (35,-25.95) node [left] {\begin{tabular}{c}$\#,1/1$\\$\#,Z_0/Z_0$\end{tabular}};
\draw [black] (31.63,-38.567) arc (-76.44684:-103.55316:26.37);
\fill [black] (19.27,-38.57) -- (19.93,-39.24) -- (20.17,-38.27);
\draw (25.45,-39.8) node [below] {$2,1/\epsilon$};
\draw [black] (19.199,-36.628) arc (106.95174:73.04826:21.439);
\fill [black] (31.7,-36.63) -- (31.08,-35.92) -- (30.79,-36.87);
\draw (25.45,-35.2) node [above] {$\epsilon,1/\epsilon$};
\draw [black] (35.823,-40.38) arc (54:-234:2.25);
\draw (34.5,-44.95) node [below] {\begin{tabular}{c}$1,1/\epsilon$\\$0,1/1$\\$0,Z_0/Z_0$\end{tabular}};
\fill [black] (33.18,-40.38) -- (32.3,-40.73) -- (33.11,-41.32);
\draw [black] (37.5,-37.7) -- (49.2,-37.7);
\fill [black] (49.2,-37.7) -- (48.4,-37.2) -- (48.4,-38.2);
\draw (43.35,-37.2) node [above] {$\epsilon,Z_0/\epsilon$};
\end{tikzpicture}
\end{center}

\newpage
\section{3. príklad}
\red{$L = \{ w_1\#w_2\ |\ w_1,w_2 \in \Sigma^*, \#_1(w_1) + (2 * \#_2(w_1)) = \#_1(w_2) + (2 * \#_2(w_2))\}$}

\section{4. príklad}

\underline{\textbf{\large{ALGORITMUS}}}\\[0.3em]
\textbf{Vstup}: Pravá lineárna gramatika $G_P = (N,\Sigma,P,S)$\\
\textbf{Výstup}: Ľavá lineárna gramatika $G_L = (N',\Sigma',P',S')$ taká, že $L(G_P)=L(G_L)$\\
\textbf{Metóda}:\\[-2em]
\begin{displayquote}
    \begin{enumerate}[1.)]
        \item pridáme pravidlo $S_0 \rightarrow S$  ak sa $S$ vyskytuje napravo
        \item $A   \rightarrow pB \Longleftrightarrow B   \rightarrow Ap$
        \item $S_P \rightarrow p  \Longleftrightarrow S_P \rightarrow p $ kde $S_P$ je počiatočný neterminál
        \item $S_P \rightarrow pA \Longleftrightarrow A   \rightarrow p $ kde $S_P$ je počiatočný neterminál
        \item $A   \rightarrow p  \Longleftrightarrow S_P \rightarrow Ap$ kde $S_P$ je počiatočný neterminál
    \end{enumerate}
\end{displayquote}

\hfill

\underline{\textbf{\large{DEMONŠTRÁCIA}}}\\[0.3em]
\textbf{Vstup}: Pravá lineárna gramatika $G = (\{S,A,B\}, \{a, b\}, P, S)$\\[-1.25em]
\begin{center}
\begin{tabular}{ll}
P:&$S \rightarrow abA\ |\ bS$\\
&$A \rightarrow bB\ |\ S\ |\ ab$\\
&$B \rightarrow \epsilon\ |\ aA$\\
\end{tabular}
\end{center}
\textbf{Výstup}: Ľavá lineárna gramatika $G_L = (N',\Sigma',P',S')$ taká, že $L(G)=L(G_L)$\\


\section{5. príklad}
\red{$L = \{ w \in \{a,b\}^*\ |\ \#_a(w)mod\ 3 \neq 0 \wedge \#_b(w) > 0\}$}
\newpage

\section{Literatúra}
\bibliographystyle{slovakiso}
\begin{flushleft}
    \bibliography{quotation}
\end{flushleft}

\end{document}

%%%%%%%%%%%%%%%%%%%%%%%%%%%%%%%%%%%%%%%%%%%%%%%%%%%%%%%%%%%%%%%%%%%%%%
% Date:
%     21.10.2018
%
% Course:
%     TIN - Intelligent Sensors
%     https://www.fit.vutbr.cz/study/courses/index.php.en?id=12941
%
% Project:
%     1. homework
%
% Authors:
%     Adrián Tóth, xtotha01@stud.fit.vutbr.cz
%%%%%%%%%%%%%%%%%%%%%%%%%%%%%%%%%%%%%%%%%%%%%%%%%%%%%%%%%%%%%%%%%%%%%%

\documentclass[11pt,a4paper]{article}

\usepackage[left=2cm,text={17cm,24cm},top=3cm]{geometry}
\usepackage[slovak]{babel}
\usepackage[utf8]{inputenc}
\usepackage[T1]{fontenc}

\usepackage{url}
\usepackage{enumerate}

\usepackage{float}
\usepackage{xcolor}
\usepackage{siunitx}
\usepackage{listings}
\usepackage{csquotes}
\usepackage{hyperref}
\usepackage{textcomp}
\usepackage{breakurl}
\usepackage{etoolbox}
\usepackage{graphicx}
\usepackage{multicol}
\usepackage{multirow}
\usepackage{supertabular}
\usepackage{tikz}
\usepackage[titles]{tocloft}

\renewcommand{\cftdot}{}
\newcommand{\red}[1]{\textcolor{red}{#1}}

\setlength\parindent{0pt}

\def\UrlBreaks{\do\/\do-}
\newcommand{\tilda}{\raisebox{0.5ex}{\texttildelow}}

\graphicspath{{.}}
\patchcmd{\thebibliography}{\section*{\refname}}{}{}{}

\begin{document}

\begin{titlepage}
    \begin{center}
        \Huge
        \textsc{
            Fakulta informačních technologií\\
            Vysoké učení technické v~Brně
        }
        \vspace{80px}
        \begin{figure}[!h]
            \centering
            \includegraphics[scale=0.3]{img/vutbr-fit-logo.eps}
        \end{figure}
        \\[15mm]
        \Huge{
            \textbf{
                TIN
            }
        }
        \\[1.5mm]
        \huge{
            \textbf{
                Teoretická informatika
            }
        }
        \\[2.5em]
        \LARGE{
            \textbf{
                2. domáca úloha
            }
        }
        \vfill
    \end{center}
        \Large{
            Adrián Tóth (xtotha01)\hfill \today
        }
\end{titlepage}

\setlength{\parskip}{0pt}
\hypersetup{hidelinks}\tableofcontents
\setlength{\parskip}{0pt}


\newpage
\section{Príklad číslo 1} %############################################################################

...

\newpage
\section{Príklad číslo 2} %############################################################################

\textit{Veta 4.19}~\cite{TIN}(str. č. 92): Nechť $L$ je bezkontextový jazyk. Pak existuje konstanta $k>0$ taková že je-li $z \in L$ a $|z| \geq k$, pak lze $z$ napsat ve tvaru:

\begin{center}
$z = uvwxy, vx \neq \varepsilon, |vwx| \leq k$
\end{center}

a pro všechna $i \geq 0$ je $uv^{i}wx^{i}y \in L$.

\rule{17cm}{0.4pt}

\hfill\\[-2em]

Nech $L_{primes}$ je bezkontextový jazyk.\\

Tak existuje celočíselná konštanta $k > 0$ taká, že ak $z \in L$ a $|z| \geq k$, tak

\begin{center}
$z = uvwxy \wedge vx \neq \varepsilon \wedge |vwx| \leq k \wedge uv^{i}wx^{i}y \in L$ kde $i \geq 0$\\
\end{center}

Zvoľme prvočíslo $r$ väčšie ako ako $k$ t.j. $r \geq k$ kde $r$ je prvočíslo.\\

Potom platí, že

\begin{center}
$a^{r} \in L \wedge |a^{r}| = r$ kde $r \geq p \ \Longrightarrow \ a^{r} = uvwxy \wedge vx \neq \varepsilon \wedge |vwx| \leq k \wedge uv^{i}wx^{i}y \in L$ pre $i \geq 0$\\
\end{center}

Nech
\begin{center}
$v = a^{M} \Rightarrow |v| = M$\\
$x = a^{N} \Rightarrow |x| = N$\\
$w = a^{O} \Rightarrow |w| = O$\\
\end{center}

Tak musí platiť že $M+N > 0$ pretože $vx \neq \varepsilon$ a $k \geq M+N+O$ pretože $|vwx| \leq k$.\\

Zvoľme $i=r+1$, potom

\begin{center}
$uv^{r+1}wx^{r+1}y \in L$\\[0.5em]
$|uv^{r+1}wx^{r+1}y| = |uvwxy| + |v^{r}| + |x^{r}| = r + r \cdot M + r \cdot N = r \cdot (1+M+N)$ čo nie je prvočíšlo\\
\end{center}

A z toho vyplýva spor pretože

\begin{center}
$uv^{r+1}wx^{r+1}y \notin L$
\end{center}

Takže jazyk $L_{primes}$ nie je bezkontextový jazyk.

\newpage
\section{Príklad číslo 3} %############################################################################

...

\newpage
\section{Príklad číslo 4} %############################################################################

...


\newpage
\section{Literatúra} %#################################################################################

\bibliographystyle{slovakiso}
\begin{flushleft}
    \bibliography{quotation}
\end{flushleft}

\end{document}
